\documentclass{article}
\usepackage{titlesec}

\usepackage[backend=bibtex,citestyle=authortitle-terse,backref]{biblatex}
\addbibresource{bibliography.bib}

\makeatletter
\renewcommand\thesection{}
\renewcommand\thesubsection{\@arabic\c@section.\@arabic\c@subsection}
\makeatother
\usepackage{tocloft}
\renewcommand{\thesubsection}{\arabic{subsection}}


\usepackage[left=4cm, right=4cm]{geometry}
\usepackage{palatino,eulervm,dutchcal,xcolor}
\usepackage{graphicx,subcaption,float}
\usepackage{enumitem,parskip,multicol,array}
\newcolumntype{L}{>{$}c<{$}}%table in mathmode
\usepackage{amsthm,amssymb,amsmath,mathtools,thmtools,tikz,tikz-cd}
\usetikzlibrary{%
	matrix,%
	calc,%
	arrows,%
	shapes,
	decorations.markings,backgrounds,calc,intersections}
\tikzcdset{scale cd/.style={every label/.append style={scale=#1},
		cells={nodes={scale=#1}}}}
\usepackage[bookmarks,bookmarksopen,bookmarksdepth=3]{hyperref}
\hypersetup{
	colorlinks=true,
	urlcolor=blue,
	linkcolor=magenta,
	citecolor=blue,
	filecolor=blue,
	urlbordercolor=white,
	linkbordercolor=white,
	citebordercolor=white,
	filebordercolor=white}
\usepackage{cleveref}

\definecolor{blue-violet}{rgb}{0.54, 0.17, 0.89}
\definecolor{azure}{rgb}{0.0, 0.5, 1.0}
\definecolor{green(ncs)}{rgb}{0.0, 0.62, 0.42}
\definecolor{forestgreen}{rgb}{0.13, 0.55, 0.13}
\definecolor{limegreen}{rgb}{0.2, 0.8, 0.2}
\definecolor{palatinateblue}{rgb}{0.15, 0.23, 0.89}
\definecolor{trueblue}{rgb}{0.0, 0.45, 0.81}
\definecolor{goldenyellow}{rgb}{1.0, 0.87, 0.0}
\definecolor{fashionfuchsia}{rgb}{0.96, 0.0, 0.63}
\definecolor{brightcerulean}{rgb}{0.11, 0.67, 0.84}
\definecolor{jonquil}{rgb}{0.98, 0.85, 0.37}
\definecolor{lavendermagenta}{rgb}{0.93, 0.51, 0.93}
\definecolor{peru}{rgb}{0.8, 0.52, 0.25}
\definecolor{persimmon}{rgb}{0.93, 0.35, 0.0}
\definecolor{persianred}{rgb}{0.8, 0.2, 0.2}
\definecolor{persianblue}{rgb}{0.11, 0.22, 0.73}
\definecolor{persiangreen}{rgb}{0.0, 0.65, 0.58}
\definecolor{persianyellow}{rgb}{0.9, 0.89, 0.0}



\declaretheoremstyle[headfont=\color{trueblue}\normalfont\bfseries,]{colored1}
\declaretheoremstyle[headfont=\color{forestgreen}\normalfont\bfseries,]{colored2}
\declaretheoremstyle[headfont=\color{peru}\normalfont\bfseries,]{colored3}
\declaretheoremstyle[headfont=\color{persiangreen}\normalfont\bfseries,]{colored4}
\declaretheoremstyle[headfont=\color{brightcerulean}\normalfont\bfseries,]{colored5}
\declaretheoremstyle[headfont=\color{lavendermagenta}\normalfont\bfseries,]{colored6}
\declaretheoremstyle[headfont=\color{blue-violet}\normalfont\bfseries,]{colored7}
\declaretheoremstyle[headfont=\color{green(ncs)}\normalfont\bfseries,]{colored8}
\declaretheoremstyle[headfont=\color{peru}\normalfont\bfseries,]{colored9}
\declaretheoremstyle[headfont=\color{persiangreen}\normalfont\bfseries,]{colored10}

\declaretheorem[style=colored1,numbered=no,name=Theorem]{thm}
\declaretheorem[style=colored2,numbered=no,name=proposition]{prop}
\declaretheorem[style=colored3,numbered=no,name=Lemma]{lemma}
\declaretheorem[style=colored4,numbered=no,name=Corollary]{coro}
\declaretheorem[style=colored5,numbered=no,name=Example]{example}
\declaretheorem[style=colored5,numbered=no,name=Examples]{exemplos}
\declaretheorem[style=colored7,numbered=no,name=Exercise]{exercise}
\declaretheorem[style=colored6,numbered=no,name=Remark]{remark}
\declaretheorem[style=colored8,numbered=no,name=Claim]{claim}
\declaretheorem[style=colored9,numbered=no,name=Definition]{defn}
\declaretheorem[style=colored10,numbered=no,name=Question]{question}

\numberwithin{equation}{section}

\newcommand{\A}{\mathbb{A}}
\newcommand{\R}{\mathbb{R}}
\newcommand{\Z}{\mathbb{Z}}
\newcommand{\N}{\mathbb{N}}
\newcommand{\C}{\mathbb{C}}
\newcommand{\Q}{\mathbb{Q}}
\newcommand{\D}{\mathbb{D}}
\renewcommand{\P}{\mathbb{P}}

\newcommand{\Ac}{\mathcal{A}}
\newcommand{\Bc}{\mathcal{B}}
\newcommand{\Cc}{\mathcal{C}}
\newcommand{\Dc}{\mathcal{D}}
\newcommand{\Ec}{\mathcal{E}}
\newcommand{\Fc}{\mathcal{F}}
\newcommand{\Gc}{\mathcal{G}}
\newcommand{\Hc}{\mathcal{H}}
\newcommand{\Ic}{\mathcal{I}}
\newcommand{\Jc}{\mathcal{J}}
\newcommand{\Kc}{\mathcal{K}}
\newcommand{\Lc}{\mathcal{L}}
\newcommand{\Mc}{\mathcal{M}}
\newcommand{\Nc}{\mathcal{N}}
\newcommand{\Oc}{\mathcal{O}}
\newcommand{\Pc}{\mathcal{P}}
\newcommand{\Qc}{\mathcal{Q}}
\newcommand{\Rc}{\mathcal{R}}
\newcommand{\Sc}{\mathcal{S}}
\newcommand{\Tc}{\mathcal{T}}
\newcommand{\Uc}{\mathcal{U}}
\newcommand{\Vc}{\mathcal{V}}
\newcommand{\Wc}{\mathcal{W}}
\newcommand{\Xc}{\mathcal{X}}
\newcommand{\Yc}{\mathcal{Y}}
\newcommand{\Zc}{\mathcal{Z}}
\newcommand{\Xf}{\mathfrak{X}}

\DeclareMathOperator{\img}{img}
\DeclareMathOperator{\Arg}{Arg}
\DeclareMathOperator{\id}{id}
\DeclareMathOperator{\pt}{pt}
\DeclareMathOperator{\Alt}{Alt}
\DeclareMathOperator{\Def}{Def}
\DeclareMathOperator{\sgn}{sgn}
\DeclareMathOperator{\hTop}{h-Top}
\DeclareMathOperator{\supp}{supp}
\DeclareMathOperator{\Int}{Int}
\DeclareMathOperator{\Ob}{Ob}
\DeclareMathOperator{\Mor}{Mor}
\DeclareMathOperator{\Top}{Top}
\DeclareMathOperator{\Set}{Set}
\DeclareMathOperator{\CGWH}{CGWH}
\DeclareMathOperator{\Hom}{Hom}
\DeclareMathOperator{\Map}{Map}
\DeclareMathOperator{\Tot}{Tot}
\DeclareMathOperator{\op}{op}
\DeclareMathOperator{\ev}{ev}
\DeclareMathOperator{\hofib}{hofib}
\DeclareMathOperator{\rel}{rel}
\DeclareMathOperator{\Vol}{Vol}
\DeclareMathOperator{\Nat}{Nat}
\DeclareMathOperator{\Arr}{Arr}
\DeclareMathOperator{\Funct}{Funct}
\DeclareMathOperator{\colim}{colim}
\DeclareMathOperator{\GL}{GL}
\DeclareMathOperator{\SL}{SL}
\DeclareMathOperator{\SO}{SO}
\DeclareMathOperator{\U}{U}
\DeclareMathOperator{\SU}{SU}
\DeclareMathOperator{\Sp}{Sp}
\DeclareMathOperator{\M}{M}
\DeclareMathOperator{\Aut}{Aut}
\DeclareMathOperator{\Bl}{Bl}
\DeclareMathOperator{\PGL}{PGL}
\DeclareMathOperator{\PSL}{PSL}
\DeclareMathOperator{\Teich}{Teich}
\DeclareMathOperator{\Diff}{Diff}
\renewcommand{\index}{\operatorname{index}}
\DeclareMathOperator{\Spec}{Spec}
\renewcommand{\Def}{\operatorname{Def}}

\begin{document}
{\LARGE notes on mirror symmetry}
%\tableofcontents
\section{3 may (Alex)}
%Main reference for these notes is \cite{gross}, sect. 14.

We wish to understand
\begin{thm}[Bogomolov-Tian-Todorov]
	Any Calabi-Yau manifold has unobstructed deformations.
\end{thm}

\begin{defn}
	An \textbf{\textit{almost compelx structure}} is an endomorphism $J$…
 \end{defn}
 \begin{remark}
 	It is a fact by Borel \& Serre (1953) that the only spheres which admit an almost complex structure are $S^2$ and $S^6$.
 \end{remark}
\begin{example}
	All complex manifolds are almost complex manifolds.
\end{example}
\begin{thm}
	A necessary and sufficient condition for a $2u$-smooth manifold $M$ to admit an almost complex structure is that the group of tangent bundle of $M$ could be reduced to $\U(n)$.
\end{thm}
\begin{thm}[Newlander-Nirenberg]
	Let $(M,J)$ be an almost complex manifold. Then, the following are equivalent:
	\begin{enumerate}
		\item (six conditions…)
	\end{enumerate}
\end{thm}
\begin{prop}
	An almost complex structure on a real 2-dimensional manifold is a complex structure.
\end{prop}
\begin{proof}
	By the Newlander-Nirenberg theorem, given a point $p\in U\subset M$ and a vector field $\Xf{U}$, we have that $(V,JV)$ is a frame, and
	\[N(V,JV)=[V,JV]+J[V,JV]+J[V,J^2V]-[JV,J^2V]=0\]
\end{proof}
\begin{defn}
	A \textbf{\textit{deformation}} of complex analytic space $M$ over a germ $(S,s_0)$ of complex analytic space is a triple $\pi_,X,i)$ such that
	\[\begin{tikzcd}
		X\arrow[d,"\pi",swap]&M\arrow[l,"\text{embedding}",swap]\arrow[d]\\
		(S,s_0)&\pt\arrow[l,"s_0"]
	\end{tikzcd}\]
	where $M$ is a compact manifold, $M\simeq\pi^{-1}(s_0)$ and $\pi$ is proper smooth.
\end{defn}
\begin{thm}[Ehresmann]
	Let $\pi:X\to S$ be a proper family of differentiable manifold. If $S$ is conncected, then all fibres are diffeomorphic.
\end{thm}
\begin{thm}[Kodaira]
	Let $X_0$ be a compact Kähler manifold. If $X\to S$ is a deformation, then any fibre $X_t$ is again Kähler.
\end{thm}
\begin{thm}[Kuranashi]\leavevmode
	\begin{enumerate}
		\item Any compact complex manifold admits a universal deformation.
		\item If $\Gamma(X_0,T_{x_0})=0$ then it admits a universal deformation.
	\end{enumerate}
\end{thm}
\begin{lemma}
	Let $J$ be an almost complex structure sufficiently close to $J_0$ so that it is represented by a form $\lambda\in A^{0,1}T^{1,0}M$. Then $J$ is integrable if and only if 
	\[\bar{\partial}\lambda_i+\frac{1}{2}[\lambda_j,\lambda_j]=0.\]
\end{lemma}
\begin{thm}[Maurer-Cartan]
	\[\bar{\partial}\phi+[\phi,\phi]=0\]
	where
	\[\phi=\phi(t)=\sum_{i=1}\phi_it_i\]
\end{thm}
\begin{defn}\leavevmode
	\begin{itemize}
		\item The \textbf{\textit{Kodaira-Spencer class}} of a one-parameter deformation $J_t$ of a complex stucture $J$ is induced by a homology class $\phi_1\in H^1(X,Tx)$.
		\item The \textbf{\textit{Kodaira-Spancer map}} is
		\begin{align*}
			T_sS&\to H^1(X_s,T_{X_s})=T_{[X_s]}\Def(X_{s_0})
		\end{align*}
	\end{itemize}
\end{defn}

\section{10 May}
\subsection{Sergey: preliminaries}
We work in the category of schemes over $\C$.
\begin{defn}
	A morphism $f:X\to Y$ is \textbf{\textit{projective}} if
	\[\begin{tikzcd}[column sep=small]
		X\arrow[rr,"f"]\arrow[dr,hook,"i",swap]&&Y\\
		&\P^n_Y=Y\times\P^n\arrow[ur]
	\end{tikzcd}\]
	where $i$ is a closed embedding and in fact $Y\times\P^n=\Spec\C$.
\end{defn}
\begin{defn}
	A \textbf{\textit{Hilbert function}} for a given $Z\hookrightarrow\P^N$ is
	\[h_Z(n)=\chi(Z,\Oc_Z(n))\]
\end{defn}
\begin{defn}[Found later in \cite{huybrechts}, p. 273]
	A morphism $f:(X,\Oc_X)\to(Y,\Oc_Y)$ if \textbf{\textit{flat}} if the stalk $\Oc_{X,x}$ {\color{magenta}[…]}
\end{defn}
\begin{claim}[Criterium for flatness of projective morphisms]
	A projective morhphism is flat if and only if $h_{X_t}(n)$ is constant {\color{magenta}as a function of $t$?
		
	$Y\to\Q[n]$}.
\end{claim}
\begin{example}[non-flat projective morphism (blowup), which is also a non-submersion]
	Let's find some $f:X\to Y$ projective but not flat. Suppose $X,Y$ are smooth and connected.
	
	A closed embedding.
		
		We tried
		\[\begin{tikzcd}[column sep=small]
			&\C^2\times\P^1\arrow[dr]\\
			X=\Bl_0\C\arrow[ur,hook]\arrow[rr,"\pi"]&&Y=\C^2
		\end{tikzcd}\]
		but {\color{magenta}(I think)} its differential is not surjective due to the tangent space of the exceptional divisor.
		
\begin{defn}[\href{https://en.wikipedia.org/wiki/Smooth_morphism}{Wiki}]
	In algebraic geometry, a morphism $f:X\to S$ between schemes is said to be \textbf{\textit{smooth}} if
	\begin{enumerate}
		\item it is locally of finite presentation.
		\item it is flat, and
		\item for every geometric point $\bar{s}\to S$ the fiber $X_{\bar{s}}=X\times_X\bar{s}$.
	\end{enumerate}
\end{defn}
	
	
	
\end{example}
\subsection{Bruno: more on deformation}
\begin{defn}[of smooth submersion] A map whose differential is surjective.
\end{defn}
\begin{defn}[\cite{gross}]
	A \textbf{\textit{deformation}} $X$ consists of a smooth proper morphism $\Xc \to S$, where $\Xc$ and $S$ are connected complex spaces, and an isomorphism $X \cong \Xc_0$, where $0\in S$ is a distinguished point. We call $\Xc\to S$ a \textbf{\textit{family of complex manifolds}}.
\end{defn}

In order to define the deformation space $\Def(X)$ suppose $X$ is Kähler with $H^0(X,\Tc_X)=0$. Then there exists a universal deformation:
\begin{defn}[\cite{gross}]
	A deformation $X\to(S,0)$ of $X$ is called \textbf{\textit{universal}} if any other deformation $X' \to (S',0')$ is isomorphic to the pullback under a uniquely determined morphism $\varphi:S'\to S$ with $\varphi(0')=0$.
	\[\begin{tikzcd}
		\Xc_S\arrow[d,"\pi_S",swap]\arrow[r]&\Xc\arrow[d]\\
		S\arrow[r,"\exists!",swap]&\Def_S(X)
	\end{tikzcd}\]
\end{defn}
\begin{defn}
	The \textbf{\textit{Teichmüller space}} of $X$ is
	\[\Teich(X)=\frac{\text{complex structures on }M}{\Diff_0}\]
	and it is such that
	\[\Tc_X\Teich(X)=H^1(X,\Tc_IX^{1,0})\]
\end{defn}
\begin{remark}[The Misha Verbitsky way]
	Let $X=(M,I)$ and $\bar\partial:C^\infty(M)\to\Omega^1(M,\C)$ and remember that
	\begin{itemize}
		\item $\img\bar\partial=\Omega^{0,1}_{(I)}(M)$
		\item $\bar\partial^2=0$.
	\end{itemize}
	Take a solution of the Maurer-Carten equation:
	\[\bar\partial\gamma+[\gamma,\gamma]=0\]
	where $\gamma\in T^{1,0}\otimes\Omega^{0,1}$. Then we do
	\begin{align*}
		(\bar\partial+\gamma)(\bar\partial f+\gamma f)&=\bar\partial(\gamma f)+\gamma\bar\partial f+\gamma(\gamma f)\\
		\bar\partial_{\text{new}}f&=\bar\partial f+\gamma f.\\
		{\color{magenta}…?}
	\end{align*}
\end{remark}
\vspace{2em}
Now take $s\in T^{1,0}\otimes\Omega^{0,1}$ such that
\[\bar\partial s+[s,s]=0\]
and consider also its cohomology class $[s]\in H^1(T^{1,0})$. We have the \textbf{\textit{Kodaira-Spencer map}}
\begin{align*}
	\operatorname{KS}:T_{s_0}S&\to H^1(T^{1,0})\cong T_X\Def X\\
	s&\mapsto [s]
\end{align*}
which is useful because de \textbf{\textit{deformation space}} of $X$ is
\[(\Def X,0)=\frac{\text{solutions to Maurer-Cartan}}{\Diff_0}\]

Ok, but what is the bracket? Answer: take the usual vector field conmutator on vector fields and the wedge product on differential forms. This makes $(\Tc^{1,0}_X\otimes \Omega^{0,\bullet}_X,[,],\bar\partial)$ into a \textbf{\textit{differential graded Lie algebra (DGLA)}}.

So suppose
\[s=\sum_{n\geq1}t^ms_m\]
and we wish to find
\[\bar\partial s_1=0\qquad\qquad\text{and}\qquad\qquad \bar\partial s_n=\sum_{i+j=n-1}[s_i,s_j]\]
The right-hand-side equation says $s_n$ is $\bar\partial$-exact.

Now since our objective is to understand Bogomolov-Tian-Todorov, we are interested in what \textbf{\textit{unobsturctedness}} is. It means that
\[\bar\partial s_1=0\qquad\qquad\text{and}\qquad\qquad\bar\partial s_2=[s_1,s_2]\]

Also recall that
\begin{defn}
	Two manifolds $M_1,M_2\subseteq\C^n$ define the same \textbf{\textit{germ}} at $0\in\C^n$ if there is an open set $U\subseteq \C^n$ containing $0$ such that
	\[M_1\cap U=M_\cap U.\]
\end{defn}
and then…
\begin{thm}[Bogomolov-Tian-Todorov]
	content...
\end{thm}

\subsection{Griffiths transversality (Victor)}
\begin{claim}
	Let $X$ be a complex manifold. For a 1-parameter family of complex structures $(X,J_t)$ and forms $\alpha_t\in\Omega^{p,q}(X,J_T)$ we have
\[\frac{d}{dt}\Big|_{t=0}\alpha_t\in\Omega^{p+1,q-1}(X)\oplus\Omega^{p,q}(X)\oplus\Omega^{p-1,q+1}(X).\]
\end{claim}
\begin{proof}
	{\color{magenta}content...}
\end{proof}

\subsection{Hodge Theory for Calabi-Yau (afternoon)}
\subsubsection{Preliminaries (Sergey)}
Let's first recall that
\begin{defn}
	The \textbf{\textit{Hodge star}} operator is 
	\begin{align*}
		*:H_\partial^{p,q}&\to H_{\bar\partial}^{n-q,n-p}\\
	\end{align*}
\end{defn}
\begin{prop}
	For any complex manifold,
	\begin{align*}
		H^{p,q}_{\bar\partial}\times H^{n-p,n-q}_{\bar\partial}&\to H^{n,n}_{\bar\partial}\cong\C\\
		([\alpha],[\beta])&\mapsto\int_{[X]}\alpha\wedge\beta:=(\alpha,\beta)
	\end{align*}
	is bilinear and non-degenerate.
\end{prop}
\begin{proof}
	$\forall\alpha\exists\beta=*\bar\alpha$ such that $(\alpha,\beta)\neq0$ so
	\begin{align*}
		0<\|\alpha\|^2=\int\alpha\wedge*\bar\alpha\\
		\langle\alpha,\beta\rangle=\int\alpha\wedge*\bar\beta
	\end{align*}
	{\color{magenta}where $\langle-,-\rangle$ is the induced metric by some hermitian/riemannian metric on $X$?}
\end{proof}

\begin{thm}[Serre duality]
	For any complex manifold,
	\begin{align*}
		H^{p,q}_{\bar\partial}\times H^{n-p,n-q}_{\bar\partial}&\to H^{n,n}_{\bar\partial}\cong\C\\
		([\alpha],[\beta])&\mapsto\int_{[X]}\alpha\wedge\beta
	\end{align*}
	is a perfect pairing. That is
	\[H^{p,q}\cong(H^{n-p,n-q})^*\]
\end{thm}
And we also have
\begin{thm}[Hodge]
	For Kähler manifolds
	\[H^{p,q}\cong \overline{H^{q,p}}\]
\end{thm}
\subsubsection{Pseudoholomorphic curves (Victor)}
We follow \cite{aroux}, lecture 3.
\begin{remark}
	For every Calabi-Yau manifold $X$,
	\[H^{p,0}= H^{n,n-p}=H^{n-p}_{\bar\partial}(X,\Omega^n_X)=H_{\bar\partial}^{n-p}(X,\Omega_X)=H^{0,n-p}=H^{n-p,0}\]
\end{remark}
So we have some symmetry:
\[\begin{tikzcd}[column sep=tiny,row sep=tiny]
	&&&1&&&&\\
	&&0&&0&&\\
	&0&&a&&0\\
	1&&b&&b&&1\\
	&0&&a&&0\\
	&&0&&0\\
	&&&1
\end{tikzcd}\qquad\qquad
\begin{tikzcd}[column sep=tiny,row sep=tiny]
	&&&1&&&&\\
	&&0&&0&&\\
	&0&&b&&0\\
	1&&a&&a&&1\\
	&0&&b&&0\\
	&&0&&0\\
	&&&1
\end{tikzcd}\]
\begin{defn}
	Let $(X^{2n},\omega)$ be a symplectic manifold, $J$ a compatible almost-complex structure, $\omega(\cdot,J\cdot)$ the associated Riemannian metric. Furthermore, let $(\Sigma, j)$ be a Riemann surface of genus $g$ and $z_1,\ldots,z_k$ marked points.

	There is a well-defined moduli space of $\Mc_{g,k}=\{(\Sigma,z_1,\ldots,z_k)\}$ which is a complex manifold of dimension $3k-3+k$.



	$u:\Sigma\to X$ is a \textbf{\textit{$J$-holomorphic (or pseudoholomorphic) map}} if
	\[J\circ du=du\circ j\]
	that is,
	\begin{equation}\label{eq:CR}
		\bar\partial_Ju=\frac{1}{2}(du+Jduj)=0.
	\end{equation}
	For $\beta\in H_2(X,\Z)$, we obtain an associated space
	\[M_{g,k}(X,J,\beta)=\{(\Sigma,j,z_1,\ldots,z_k,u:\Sigma\to X|u_*[\Sigma]=\beta,\bar\partial_Ju=0\}/\sim\]
	where $\sim$ is the equivalence given by $\phi$ below:
	\[\begin{tikzcd}
		\Sigma,z_1,\ldots,z_k\arrow[r,"u"]\arrow[d,"\phi",swap]\arrow[d,"\cong"]&X\\
		\Sigma',z_1',\ldots,z_k'\arrow[ur,"u'",swap]
	\end{tikzcd}\]
\end{defn}
\begin{question}
	Where does the object in \cref{eq:CR} live? The differential of any map of complex manifolds can be decomposed in $\partial$ and $\bar\partial$. The operator $\bar\partial_Ju$ is an element of $\Omega^{0,1}(\Sigma,u^* TX)=\Gamma(\Sigma,\Omega^{0,1}(\Sigma)\otimes u^*TX)$.
\end{question}
\begin{remark}
	See \href{https://en.wikipedia.org/wiki/Pseudoholomorphic_curve#Analogy_with_the_classical_Cauchy–Riemann_equations}{wiki} for interpretation of this definition as a map satisfying the Cauchy-Riemann equations.
\end{remark}
\begin{remark}
	See \href{http://members.unine.ch/felix.schlenk/Maths/What/pseudoholomorphic.pdf}{What is... a pseudoholomorphic curve?} for another friendly explanation:
	\begin{quotation}
		A pseudoholomorphic curve is just the natural modification of the notion of a holomorphic curve to the case when the ambient manifold is almost-complex.
	\end{quotation}
\end{remark}

\section{May 17}
\subsection{Pseudoholomorphic curves cont. (Victor)}
We continue to read \cite{aroux}, lecture 3.
\begin{defn}
	We say that $u:\Sigma\to X$ is \textbf{\textit{simple}} if there exists $z\in\Sigma$ such that $du(z)\neq0$ and $u^{-1}(u(z))={z}$.
\end{defn}
Which roughly means that the function is not generically one to one on its image.
\begin{example}
	The function
	\begin{align*}
		u:\P^1&\to\P^2\\
		[x:y]&\mapsto[x^2:y^2:0]
	\end{align*}
	is not simple. Indeed, near a point $[x:y]\in\P^1$ with $x\neq0$, the differential of $u$ may be expressed in coordinates as the linear map $du=\begin{pmatrix}
		2&0
	\end{pmatrix}\neq0$; however $u^{-1}([x^2:y^2:0])=\{[x:y],[-x:y]\}$. The case of $y\neq0$ is analogous. We also see there are no singular points, so $u$ cannot be simple.
\end{example}
Then we define
\begin{align*}
	D_{\bar\partial}:W^{r+1,p}(\Sigma,u^*TX)\times T\Mc_{g,k}\to W^{r,p}(\Sigma,\Omega^{0,1}_\Sigma\otimes U^*TX)
\end{align*}
by 
\[D_{\bar\partial}(v,j')=\bar\partial v+\frac{1}{2}(\nabla_vJ)du\cdot j+\frac{1}{2}J\cdot du\cdot j'\]
where $W^{r,p}$ is a completion of $C^\infty(-)$ of (?) to $L^{r,p}$ norm defined by $\|f\|_{r,p}=\left(\sum_{i=0}^r\int|f^{(i)}(t)|^pdt\right)^{1/p}$.

$D_{\bar\partial}$ is Fredholm, (meaning the dimensions of its kernel and cokernel are finite), with index (the difference of such numbers)
\[\index_\R D_{\bar\partial}:=2d=2\langle c_1(TX),\beta\rangle+n(2-2g)+(6g-6+2k).\]

We may interpret this equation as differentiation of the Cauchy-Riemann equations.

\subsection{Dirichlet energy functional (Alex)}
	We follow \cite{mcduff}, sec. 2.2
	
	Consider a map
	\[u:(\Sigma,j)\to(X,\omega,J,g)\]
	and define the \textbf{\textit{energy functional}}
	\[\varepsilon=\int_\Sigma|du|^2_g\Vol_g\]
	Now, we may take local isothermic coordinates where the metric is expressed as
	\[g=\lambda(x,y)(dx^2+dy^2)\]
	giving
	\begin{align*}
		du&=\partial_xu\otimes dx+\partial_yu\otimes dy\\
		|du|^2&=|\partial_xu|^2\lambda^{-2}+|\partial_yu|^2\lambda^{/2}\\
		\Vol_{\Sigma g'}&=\lambda^2dx\wedge dy
	\end{align*}
	Then
	\[\varepsilon(u)=\int_\Sigma|\partial_xu|^2_g+|\partial_yu|_g^2dx\wedge dy.\]
	The following equality shows that the energy functional attains its minimum on pseudoholomorphic maps (in virtue of \cref{eq:CR}).
	\begin{claim}
		For a pseudoholomorphic map,
	\[\varepsilon(u)=\int_\Sigma2|\bar\partial_J|^2\Vol+\int_\Sigma u^*\omega\]
	\end{claim}
	\begin{proof}
		content...
	\end{proof}

\section{24 may (Alex)}
We start with two short questions from last session.
\begin{question}\leavevmode
	\begin{itemize}
		\item What exactly is $\Omega^1(\Sigma,E)$ where $E$ is a vector bundle? It is the space of sections of the bundle $T^*M\Sigma\otimes E$.
		
		\item Let $u:(\Sigma,j)\to(X,J)$. Is $du$ is an element of $\Omega^{1}(\Sigma,u^*TM)$? Yes, notice that $du$ is an element of $\Hom(T\Sigma,TX)$. Forget about all of $TX$ and consider only its image under $u$. There is an isomorphism $T\Sigma^*\otimes u^*TX\cong \Hom(T\Sigma,u^*TX)$.
	\end{itemize}
\end{question}
\begin{remark}
	There is a bundle $\Ec\to\Bc$ where $\Bc=C^\infty(\Sigma,M)$ and the fibers are $\Ec_u=\Omega^{0,1}(\Sigma,u^*TM)$. For a map $u:(\Sigma,j)\to(X,J)$, the nonlinear operator … [quote from \cite{mcduff}].
\end{remark}
Then we concluded the proof of the final claim of the last session.
\addcontentsline{toc}{section}{References}
\printbibliography
%\clearpage
\end{document}
