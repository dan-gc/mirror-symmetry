\documentclass{article}
\usepackage{titlesec}

\usepackage[citestyle=authortitle-terse,backref,backend=bibtex]{biblatex}
\addbibresource{bibliography.bib}

\makeatletter
\renewcommand\thesection{}
\renewcommand\thesubsection{\@arabic\c@section.\@arabic\c@subsection}
\makeatother
\usepackage{tocloft}
\renewcommand{\thesubsection}{\arabic{subsection}}


\usepackage[left=4cm, right=4cm]{geometry}
\usepackage{palatino,eulervm,dutchcal,xcolor}
\usepackage{graphicx,subcaption,float}
\usepackage{enumitem,parskip,multicol,array}
\newcolumntype{L}{>{$}c<{$}}%table in mathmode
\usepackage{amsthm,amssymb,amsmath,mathtools,thmtools,tikz,tikz-cd}
\usetikzlibrary{%
	matrix,%
	calc,%
	arrows,%
	shapes,
	decorations.markings,backgrounds,calc,intersections}
\tikzcdset{scale cd/.style={every label/.append style={scale=#1},
		cells={nodes={scale=#1}}}}
\usepackage[bookmarks,bookmarksopen,bookmarksdepth=3]{hyperref}
\hypersetup{%colores
	colorlinks=true,
	urlcolor=blue,
	linkcolor=magenta,
	citecolor=blue,
	filecolor=blue,
	urlbordercolor=white,
	linkbordercolor=white,
	citebordercolor=white,
	filebordercolor=white}
\usepackage{cleveref}

\definecolor{blue-violet}{rgb}{0.54, 0.17, 0.89}
\definecolor{azure}{rgb}{0.0, 0.5, 1.0}
\definecolor{green(ncs)}{rgb}{0.0, 0.62, 0.42}
\definecolor{forestgreen}{rgb}{0.13, 0.55, 0.13}
\definecolor{limegreen}{rgb}{0.2, 0.8, 0.2}
\definecolor{palatinateblue}{rgb}{0.15, 0.23, 0.89}
\definecolor{trueblue}{rgb}{0.0, 0.45, 0.81}
\definecolor{goldenyellow}{rgb}{1.0, 0.87, 0.0}
\definecolor{fashionfuchsia}{rgb}{0.96, 0.0, 0.63}
\definecolor{brightcerulean}{rgb}{0.11, 0.67, 0.84}
\definecolor{jonquil}{rgb}{0.98, 0.85, 0.37}
\definecolor{lavendermagenta}{rgb}{0.93, 0.51, 0.93}
\definecolor{peru}{rgb}{0.8, 0.52, 0.25}
\definecolor{persimmon}{rgb}{0.93, 0.35, 0.0}
\definecolor{persianred}{rgb}{0.8, 0.2, 0.2}
\definecolor{persianblue}{rgb}{0.11, 0.22, 0.73}
\definecolor{persiangreen}{rgb}{0.0, 0.65, 0.58}
\definecolor{persianyellow}{rgb}{0.9, 0.89, 0.0}



\declaretheoremstyle[headfont=\color{trueblue}\normalfont\bfseries,]{colored1}
\declaretheoremstyle[headfont=\color{forestgreen}\normalfont\bfseries,]{colored2}
\declaretheoremstyle[headfont=\color{peru}\normalfont\bfseries,]{colored3}
\declaretheoremstyle[headfont=\color{persiangreen}\normalfont\bfseries,]{colored4}
\declaretheoremstyle[headfont=\color{brightcerulean}\normalfont\bfseries,]{colored5}
\declaretheoremstyle[headfont=\color{lavendermagenta}\normalfont\bfseries,]{colored6}
\declaretheoremstyle[headfont=\color{blue-violet}\normalfont\bfseries,]{colored7}
\declaretheoremstyle[headfont=\color{green(ncs)}\normalfont\bfseries,]{colored8}
\declaretheoremstyle[headfont=\color{peru}\normalfont\bfseries,]{colored9}
\declaretheoremstyle[headfont=\color{persiangreen}\normalfont\bfseries,]{colored10}

\declaretheorem[style=colored1,name=Theorem]{thm}
\declaretheorem[style=colored2,numberlike=thm,name=proposition]{prop}
\declaretheorem[style=colored3,numbered=no,name=Lemma]{lemma}
\declaretheorem[style=colored4,numbered=no,name=Corollary]{coro}
\declaretheorem[style=colored5,numbered=no,name=Example]{example}
\declaretheorem[style=colored5,numbered=no,name=Examples]{exemplos}
\declaretheorem[style=colored7,numbered=no,name=Exercise]{exercise}
\declaretheorem[style=colored6,numbered=no,name=Remark]{remark}
\declaretheorem[style=colored8,numbered=no,name=Claim]{claim}
\declaretheorem[style=colored9,numbered=no,name=Definition]{defn}
\declaretheorem[style=colored10,numbered=no,name=Question]{question}

\numberwithin{equation}{section}

\newcommand{\A}{\mathbb{A}}
\newcommand{\R}{\mathbb{R}}
\newcommand{\Z}{\mathbb{Z}}
\newcommand{\N}{\mathbb{N}}
\newcommand{\C}{\mathbb{C}}
\newcommand{\Q}{\mathbb{Q}}
\newcommand{\D}{\mathbb{D}}
\renewcommand{\P}{\mathbb{P}}

\newcommand{\Ac}{\mathcal{A}}
\newcommand{\Bc}{\mathcal{B}}
\newcommand{\Cc}{\mathcal{C}}
\newcommand{\Dc}{\mathcal{D}}
\newcommand{\Ec}{\mathcal{E}}
\newcommand{\Fc}{\mathcal{F}}
\newcommand{\Gc}{\mathcal{G}}
\newcommand{\Hc}{\mathcal{H}}
\newcommand{\Ic}{\mathcal{I}}
\newcommand{\Jc}{\mathcal{J}}
\newcommand{\Kc}{\mathcal{K}}
\newcommand{\Lc}{\mathcal{L}}
\newcommand{\Mc}{\mathcal{M}}
\newcommand{\Nc}{\mathcal{N}}
\newcommand{\Oc}{\mathcal{O}}
\newcommand{\Pc}{\mathcal{P}}
\newcommand{\Qc}{\mathcal{Q}}
\newcommand{\Rc}{\mathcal{R}}
\newcommand{\Sc}{\mathcal{S}}
\newcommand{\Tc}{\mathcal{T}}
\newcommand{\Uc}{\mathcal{U}}
\newcommand{\Vc}{\mathcal{V}}
\newcommand{\Wc}{\mathcal{W}}
\newcommand{\Xc}{\mathcal{X}}
\newcommand{\Yc}{\mathcal{Y}}
\newcommand{\Zc}{\mathcal{Z}}
\newcommand{\Xf}{\mathfrak{X}}

\DeclareMathOperator{\img}{img}
\DeclareMathOperator{\Arg}{Arg}
\DeclareMathOperator{\id}{id}
\DeclareMathOperator{\pt}{pt}
\DeclareMathOperator{\Alt}{Alt}
\DeclareMathOperator{\Def}{Def}
\DeclareMathOperator{\sgn}{sgn}
\DeclareMathOperator{\hTop}{h-Top}
\DeclareMathOperator{\supp}{supp}
\DeclareMathOperator{\Int}{Int}
\DeclareMathOperator{\Ob}{Ob}
\DeclareMathOperator{\Mor}{Mor}
\DeclareMathOperator{\Top}{Top}
\DeclareMathOperator{\Set}{Set}
\DeclareMathOperator{\CGWH}{CGWH}
\DeclareMathOperator{\Hom}{Hom}
\DeclareMathOperator{\Map}{Map}
\DeclareMathOperator{\Tot}{Tot}
\DeclareMathOperator{\op}{op}
\DeclareMathOperator{\ev}{ev}
\DeclareMathOperator{\hofib}{hofib}
\DeclareMathOperator{\rel}{rel}
\DeclareMathOperator{\Nat}{Nat}
\DeclareMathOperator{\Arr}{Arr}
\DeclareMathOperator{\Funct}{Funct}
\DeclareMathOperator{\colim}{colim}
\DeclareMathOperator{\GL}{GL}
\DeclareMathOperator{\SL}{SL}
\DeclareMathOperator{\SO}{SO}
\DeclareMathOperator{\U}{U}
\DeclareMathOperator{\SU}{SU}
\DeclareMathOperator{\Sp}{Sp}
\DeclareMathOperator{\M}{M}
\DeclareMathOperator{\Aut}{Aut}
\DeclareMathOperator{\PGL}{PGL}
\DeclareMathOperator{\PSL}{PSL}

\begin{document}
{\LARGE notes on mirror symmetry}
%\tableofcontents
\section{3 may (Alex)}
Main reference for these notes is \cite{gross}, sect. 14.

We wish to understand
\begin{thm}[Bogomolov-Tian-Todorov]
	Any Calabi-Yau manifold has unobstructed deformations.
\end{thm}

\begin{defn}
	An \textbf{\textit{almost compelx structure}} is an endomorphism $J$…
 \end{defn}
 \begin{remark}
 	It is a fact by Borel \& Serre (1953) that the only spheres which admit an almost complex structure are $S^2$ and $S^6$.
 \end{remark}
\begin{example}
	All complex manifolds are almost complex manifolds.
\end{example}
\begin{thm}
	A necessary and sufficient condition for a $2u$-smooth manifold $M$ to admit an almost complex structure is that the group of tangent bundle of $M$ could be reduced to $\U(n)$.
\end{thm}
\begin{thm}[Newlander-Nirenberg]
	Let $(M,J)$ be an almost complex manifold. Then, the following are equivalent:
	\begin{enumerate}
		\item (six conditions…)
	\end{enumerate}
\end{thm}
\begin{prop}
	An almost complex structure on a real 2-dimensional manifold is a complex structure.
\end{prop}
\begin{proof}
	By the Newlander-Nirenberg theorem, given a point $p\in U\subset M$ and a vector field $\Xf{U}$, we have that $(V,JV)$ is a frame, and
	\[N(V,JV)=[V,JV]+J[V,JV]+J[V,J^2V]-[JV,J^2V]=0\]
\end{proof}
\begin{defn}
	A \textbf{\textit{deformation}} of complex analytic space $M$ over a germ $(S,s_0)$ of complex analytic space is a triple $\pi_,X,i)$ such that
	\[\begin{tikzcd}
		X\arrow[d,"\pi",swap]&M\arrow[l,"\text{embedding}",swap]\arrow[d]\\
		(S,s_0)&\pt\arrow[l,"s_0"]
	\end{tikzcd}\]
	where $M$ is a compact manifold, $M\simeq\pi^{-1}(s_0)$ and $\pi$ is proper smooth.
\end{defn}
\begin{thm}[Ehresmann]
	Let $\pi:X\to S$ be a proper family of differentiable manifold. If $S$ is conncected, then all fibres are diffeomorphic.
\end{thm}
\begin{thm}[Kodaira]
	Let $X_0$ be a compact Kähler manifold. If $X\to S$ is a deformation, then any fibre $X_t$ is again Kähler.
\end{thm}
\begin{thm}[Kuranashi]\leavevmode
	\begin{enumerate}
		\item Any compact complex manifold admits a universal deformation.
		\item If $\Gamma(X_0,T_{x_0})=0$ then it admits a universal deformation.
	\end{enumerate}
\end{thm}
\begin{lemma}
	Let $J$ be an almost complex structure sufficiently close to $J_0$ so that it is represented by a form $\lambda\in A^{0,1}T^{1,0}M$. Then $J$ is integrable if and only if 
	\[\bar{\partial}\lambda_i+\frac{1}{2}[\lambda_j,\lambda_j]=0.\]
\end{lemma}
\begin{thm}[Maurier-Cartan]
	\[\bar{\partial}\phi+[\phi,\phi]=0\]
	where
	\[\phi=\phi(t)=\sum_{i=1}\phi_it_i\]
\end{thm}
\begin{defn}\leavevmode
	\begin{itemize}
		\item The \textbf{\textit{Kodaira-Spencer class}} of a one-parameter deformation $J_t$ of a complex stucture $J$ is induced by a homology class $\phi_1\in H^1(X,Tx)$.
		\item The \textbf{\textit{Kodaira-Spancer map}} is
		\begin{align*}
			T_sS&\to H^1(X_s,T_{X_s})=T_{[X_s]}\Def(X_{s_0})
		\end{align*}
	\end{itemize}
\end{defn}

\clearpage
\addcontentsline{toc}{section}{References}
\printbibliography
\clearpage
\end{document}